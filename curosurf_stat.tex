% Options for packages loaded elsewhere
\PassOptionsToPackage{unicode}{hyperref}
\PassOptionsToPackage{hyphens}{url}
\PassOptionsToPackage{dvipsnames,svgnames,x11names}{xcolor}
%
\documentclass[
  10pt,
  a4paper,
]{scrartcl}
\usepackage{amsmath,amssymb}
\usepackage{lmodern}
\usepackage{iftex}
\ifPDFTeX
  \usepackage[T1]{fontenc}
  \usepackage[utf8]{inputenc}
  \usepackage{textcomp} % provide euro and other symbols
\else % if luatex or xetex
  \usepackage{unicode-math}
  \defaultfontfeatures{Scale=MatchLowercase}
  \defaultfontfeatures[\rmfamily]{Ligatures=TeX,Scale=1}
  \setmainfont[Ligatures = Common,Ligatures = Rare,Style = Swash]{Adobe
Garamond Pro}
  \setsansfont[]{Trajan Pro}
  \setmonofont[Numbers = Monospaced]{Source Sans Pro}
  \setmathfont[Numbers = Monospaced]{Source Sans Pro}
\fi
% Use upquote if available, for straight quotes in verbatim environments
\IfFileExists{upquote.sty}{\usepackage{upquote}}{}
\IfFileExists{microtype.sty}{% use microtype if available
  \usepackage[]{microtype}
  \UseMicrotypeSet[protrusion]{basicmath} % disable protrusion for tt fonts
}{}
\makeatletter
\@ifundefined{KOMAClassName}{% if non-KOMA class
  \IfFileExists{parskip.sty}{%
    \usepackage{parskip}
  }{% else
    \setlength{\parindent}{0pt}
    \setlength{\parskip}{6pt plus 2pt minus 1pt}}
}{% if KOMA class
  \KOMAoptions{parskip=half}}
\makeatother
\usepackage{xcolor}
\IfFileExists{xurl.sty}{\usepackage{xurl}}{} % add URL line breaks if available
\IfFileExists{bookmark.sty}{\usepackage{bookmark}}{\usepackage{hyperref}}
\hypersetup{
  pdftitle={Cortico Anténatale et surfactant},
  pdfauthor={Philippe MICHEL},
  pdflang={fr},
  colorlinks=true,
  linkcolor={Maroon},
  filecolor={Maroon},
  citecolor={Blue},
  urlcolor={Blue},
  pdfcreator={LaTeX via pandoc}}
\urlstyle{same} % disable monospaced font for URLs
\usepackage{graphicx}
\makeatletter
\def\maxwidth{\ifdim\Gin@nat@width>\linewidth\linewidth\else\Gin@nat@width\fi}
\def\maxheight{\ifdim\Gin@nat@height>\textheight\textheight\else\Gin@nat@height\fi}
\makeatother
% Scale images if necessary, so that they will not overflow the page
% margins by default, and it is still possible to overwrite the defaults
% using explicit options in \includegraphics[width, height, ...]{}
\setkeys{Gin}{width=\maxwidth,height=\maxheight,keepaspectratio}
% Set default figure placement to htbp
\makeatletter
\def\fps@figure{htbp}
\makeatother
\setlength{\emergencystretch}{3em} % prevent overfull lines
\providecommand{\tightlist}{%
  \setlength{\itemsep}{0pt}\setlength{\parskip}{0pt}}
\setcounter{secnumdepth}{-\maxdimen} % remove section numbering
\newlength{\cslhangindent}
\setlength{\cslhangindent}{1.5em}
\newlength{\csllabelwidth}
\setlength{\csllabelwidth}{3em}
\newlength{\cslentryspacingunit} % times entry-spacing
\setlength{\cslentryspacingunit}{\parskip}
\newenvironment{CSLReferences}[2] % #1 hanging-ident, #2 entry spacing
 {% don't indent paragraphs
  \setlength{\parindent}{0pt}
  % turn on hanging indent if param 1 is 1
  \ifodd #1
  \let\oldpar\par
  \def\par{\hangindent=\cslhangindent\oldpar}
  \fi
  % set entry spacing
  \setlength{\parskip}{#2\cslentryspacingunit}
 }%
 {}
\usepackage{calc}
\newcommand{\CSLBlock}[1]{#1\hfill\break}
\newcommand{\CSLLeftMargin}[1]{\parbox[t]{\csllabelwidth}{#1}}
\newcommand{\CSLRightInline}[1]{\parbox[t]{\linewidth - \csllabelwidth}{#1}\break}
\newcommand{\CSLIndent}[1]{\hspace{\cslhangindent}#1}
\ifLuaTeX
\usepackage[bidi=basic]{babel}
\else
\usepackage[bidi=default]{babel}
\fi
\babelprovide[main,import]{french}
% get rid of language-specific shorthands (see #6817):
\let\LanguageShortHands\languageshorthands
\def\languageshorthands#1{}
\usepackage{booktabs}
\usepackage{longtable}
\usepackage{array}
\usepackage{multirow}
\usepackage{wrapfig}
\usepackage{float}
\usepackage{colortbl}
\usepackage{pdflscape}
\usepackage{tabu}
\usepackage{threeparttable}
\usepackage{threeparttablex}
\usepackage[normalem]{ulem}
\usepackage{makecell}
\usepackage{xcolor}
\ifLuaTeX
  \usepackage{selnolig}  % disable illegal ligatures
\fi

\title{Cortico Anténatale et surfactant\thanks{Suzi MANSOUR -
Réanimation néonatale}}
\usepackage{etoolbox}
\makeatletter
\providecommand{\subtitle}[1]{% add subtitle to \maketitle
  \apptocmd{\@title}{\par {\large #1 \par}}{}{}
}
\makeatother
\subtitle{Quelques chiffres}
\author{Philippe MICHEL}
\date{17 juin 2022}

\begin{document}
\maketitle

{
\hypersetup{linkcolor=}
\setcounter{tocdepth}{2}
\tableofcontents
}
\listoffigures
\listoftables
\hypertarget{description-de-la-population}{%
\subsection{Description de la
population}\label{description-de-la-population}}

\begin{longtable}[t]{lcccc}
\caption{\label{tab:desc}Tableau descriptif}\\
\toprule
\multicolumn{2}{c}{ } & \multicolumn{2}{c}{Corticoïdes antenatal} & \multicolumn{1}{c}{ } \\
\cmidrule(l{3pt}r{3pt}){3-4}
\textbf{Characteristic} & \textbf{Overall}, N = 118 & \textbf{Demi dose}, N = 23 & \textbf{Pleine dose}, N = 95 & \textbf{p-value}\\
\midrule
\endfirsthead
\caption[]{Tableau descriptif \textit{(continued)}}\\
\toprule
\multicolumn{2}{c}{ } & \multicolumn{2}{c}{Corticoïdes antenatal} & \multicolumn{1}{c}{ } \\
\cmidrule(l{3pt}r{3pt}){3-4}
\textbf{Characteristic} & \textbf{Overall}, N = 118 & \textbf{Demi dose}, N = 23 & \textbf{Pleine dose}, N = 95 & \textbf{p-value}\\
\midrule
\endhead

\endfoot
\bottomrule
\endlastfoot
\textbf{sexe} &  &  &  & 0.2\\
\hspace{1em}F & 54 (46\%) & 8 (35\%) & 46 (48\%) & \\
\hspace{1em}H & 64 (54\%) & 15 (65\%) & 49 (52\%) & \\
\textbf{diabete.1} &  &  &  & 0.5\\
\hspace{1em}non & 115 (97\%) & 22 (96\%) & 93 (98\%) & \\
\addlinespace
\hspace{1em}oui & 3 (2.5\%) & 1 (4.3\%) & 2 (2.1\%) & \\
\textbf{diabete.gestationnel.sous.insuline} &  &  &  & >0.9\\
\hspace{1em}non & 116 (98\%) & 23 (100\%) & 93 (98\%) & \\
\hspace{1em}oui & 2 (1.7\%) & 0 (0\%) & 2 (2.1\%) & \\
\textbf{mode.de.sortie} &  &  &  & 0.4\\
\addlinespace
\hspace{1em}Décès & 23 (19\%) & 4 (17\%) & 19 (20\%) & \\
\hspace{1em}Domicile & 41 (35\%) & 11 (48\%) & 30 (32\%) & \\
\hspace{1em}Transfert & 54 (46\%) & 8 (35\%) & 46 (48\%) & \\
\textbf{intervalle.cortico.accouchement} &  &  &  & \textbf{<0.001}\\
\hspace{1em}< 12 H & 20 (17\%) & 15 (65\%) & 5 (5.3\%) & \\
\addlinespace
\hspace{1em}entre 12 H et 24H & 18 (15\%) & 8 (35\%) & 10 (11\%) & \\
\hspace{1em}> 24 H & 79 (68\%) & 0 (0\%) & 79 (84\%) & \\
\hspace{1em}Unknown & 1 & 0 & 1 & \\
\textbf{cause.prematurite} &  &  &  & 0.8\\
\hspace{1em}spontanée & 66 (56\%) & 14 (61\%) & 52 (55\%) & \\
\addlinespace
\hspace{1em}ARCF & 16 (14\%) & 3 (13\%) & 13 (14\%) & \\
\hspace{1em}Pré éclampsie & 29 (25\%) & 4 (17\%) & 25 (26\%) & \\
\hspace{1em}autre & 7 (5.9\%) & 2 (8.7\%) & 5 (5.3\%) & \\
\textbf{cut.terme} &  &  &  & 0.093\\
\hspace{1em}prématurité moyenne & 18 (15\%) & 7 (30\%) & 11 (12\%) & \\
\addlinespace
\hspace{1em}grande prématurité & 43 (36\%) & 6 (26\%) & 37 (39\%) & \\
\hspace{1em}très grande prématurité & 57 (48\%) & 10 (43\%) & 47 (49\%) & \\
\textbf{voie.dacc} &  &  &  & 0.3\\
\hspace{1em}césarienne & 67 (57\%) & 11 (48\%) & 56 (59\%) & \\
\hspace{1em}VB & 51 (43\%) & 12 (52\%) & 39 (41\%) & \\
\addlinespace
\textbf{poids.naissance} & 965 (764, 1,410) & 1,110 (835, 1,886) & 935 (728, 1,270) & 0.063\\
\textbf{audipog} & 52 (18, 75) & 63 (46, 84) & 44 (17, 69) & \textbf{0.016}\\
\textbf{rciu} &  &  &  & 0.3\\
\hspace{1em}Macrosomie & 1 (0.8\%) & 1 (4.3\%) & 0 (0\%) & \\
\hspace{1em}non & 99 (84\%) & 20 (87\%) & 79 (83\%) & \\
\addlinespace
\hspace{1em}RCIU modéré & 12 (10\%) & 2 (8.7\%) & 10 (11\%) & \\
\hspace{1em}RCIU sévère & 6 (5.1\%) & 0 (0\%) & 6 (6.3\%) & \\
\textbf{imf} &  &  &  & 0.4\\
\hspace{1em}non & 50 (42\%) & 8 (35\%) & 42 (44\%) & \\
\hspace{1em}Suspectée & 56 (47\%) & 11 (48\%) & 45 (47\%) & \\
\addlinespace
\hspace{1em}Confirmée & 12 (10\%) & 4 (17\%) & 8 (8.4\%) & \\
\textbf{reanimation.en.sdn} &  &  &  & 0.7\\
\hspace{1em}intubation & 2 (1.7\%) & 1 (4.5\%) & 1 (1.1\%) & \\
\hspace{1em}Intubation & 51 (44\%) & 9 (41\%) & 42 (44\%) & \\
\hspace{1em}Intubation+Drogues & 6 (5.1\%) & 1 (4.5\%) & 5 (5.3\%) & \\
\addlinespace
\hspace{1em}non & 4 (3.4\%) & 0 (0\%) & 4 (4.2\%) & \\
\hspace{1em}VNI & 54 (46\%) & 11 (50\%) & 43 (45\%) & \\
\hspace{1em}Unknown & 1 & 1 & 0 & \\
\textbf{p.h.au.cordon} & 7.28 (7.23, 7.32) & 7.24 (7.14, 7.34) & 7.29 (7.24, 7.32) & 0.3\\
\hspace{1em}Unknown & 11 & 5 & 6 & \\
\addlinespace
\textbf{lactates} & 3.20 (2.50, 4.90) & 4.00 (2.60, 7.20) & 3.10 (2.50, 4.40) & 0.2\\
\hspace{1em}Unknown & 13 & 6 & 7 & \\
\textbf{intubation} &  &  &  & 0.6\\
\hspace{1em}non & 4 (3.4\%) & 1 (4.5\%) & 3 (3.2\%) & \\
\hspace{1em}oui & 112 (97\%) & 21 (95\%) & 91 (97\%) & \\
\addlinespace
\hspace{1em}Unknown & 2 & 1 & 1 \vphantom{2} & \\
\textbf{nbre.intubations} &  &  &  & 0.7\\
\hspace{1em}≥ 3 & 13 (11\%) & 3 (13\%) & 10 (11\%) & \\
\hspace{1em}1 & 75 (64\%) & 16 (70\%) & 59 (62\%) & \\
\hspace{1em}2 & 30 (25\%) & 4 (17\%) & 26 (27\%) & \\
\addlinespace
\textbf{in.sur.e} &  &  &  & >0.9\\
\hspace{1em}non & 111 (96\%) & 21 (95\%) & 90 (96\%) & \\
\hspace{1em}oui & 5 (4.3\%) & 1 (4.5\%) & 4 (4.3\%) & \\
\hspace{1em}Unknown & 2 & 1 & 1 \vphantom{1} & \\
\textbf{lisa} &  &  &  & >0.9\\
\addlinespace
\hspace{1em}non & 113 (97\%) & 22 (100\%) & 91 (97\%) & \\
\hspace{1em}oui & 3 (2.6\%) & 0 (0\%) & 3 (3.2\%) & \\
\hspace{1em}Unknown & 2 & 1 & 1 & \\
\textbf{fi.o2} & 0.53 (0.40, 1.00) & 0.50 (0.45, 0.60) & 0.55 (0.40, 1.00) & 0.8\\
\hspace{1em}Unknown & 6 & 2 & 4 & \\
\addlinespace
\textbf{detresse.respiratoire} &  &  &  & >0.9\\
\hspace{1em}non & 1 (0.8\%) & 0 (0\%) & 1 (1.1\%) & \\
\hspace{1em}oui & 110 (93\%) & 22 (96\%) & 88 (93\%) & \\
\hspace{1em}secondaire & 7 (5.9\%) & 1 (4.3\%) & 6 (6.3\%) & \\*
\multicolumn{5}{l}{\rule{0pt}{1em}\textsuperscript{1} n (\%); Median (IQR)}\\
\multicolumn{5}{l}{\rule{0pt}{1em}\textsuperscript{2} Pearson's Chi-squared test; Fisher's exact test; Wilcoxon rank sum test}\\
\end{longtable}

\includegraphics{curosurf_stat_files/figure-latex/pbterme-1.pdf}

\hypertarget{crituxe8re-principal}{%
\subsection{Critère principal}\label{crituxe8re-principal}}

le critère principal est le nombre de doses de surfactant (Curosurf©)
reçues (1 dose vs deux doses ou plus) selon le traitement corticoïde
reçu avant l'accouchement (une dose vs deux doses).

\textbf{Problème} : le nombre de doses reçues en anténatal est lié au
terme \& au score AUDIPOG.

\includegraphics{curosurf_stat_files/figure-latex/principal-1.pdf}

\begin{verbatim}
## 
##                    tt$corticotherapie.an
## tt$nb.dose.curosurf Demi dose Pleine dose Total
##     1 dose                 16          72    88
##     2 doses ou plus         7          23    30
##     Total                  23          95   118
## 
## OR =  0.73 
## 95% CI =  0.27, 1.99  
## Chi-squared = 0.38, 1 d.f., P value = 0.538
## Fisher's exact test (2-sided) P value = 0.596
\end{verbatim}

Il n'y a pas de différence significative entre les deux groupes
concernant le critère principal.

\includegraphics{curosurf_stat_files/figure-latex/gprinci1-1.pdf}

\hypertarget{facteurs-explicatifs}{%
\subsubsection{Facteurs explicatifs}\label{facteurs-explicatifs}}

On recherche des facteurs autres pouvant influer sur le nombre de dose
de surfactant reçues.

\begin{table}

\caption{\label{tab:princ1}Facteurs autres}
\centering
\begin{tabular}[t]{l|c|c|c|c}
\hline
\multicolumn{2}{c|}{ } & \multicolumn{2}{c|}{Doses de surfactants} & \multicolumn{1}{c}{ } \\
\cline{3-4}
\textbf{Characteristic} & \textbf{Total} (N = 118) & \textbf{1 dose}, N = 88 & \textbf{2 doses ou plus}, N = 30 & \textbf{p-value}\\
\hline
\textbf{sexe} &  &  &  & 0.3\\
\hline
\hspace{1em}F & 54 (46\%) & 38 (43\%) & 16 (53\%) & \\
\hline
\hspace{1em}H & 64 (54\%) & 50 (57\%) & 14 (47\%) & \\
\hline
\textbf{diabete.1} &  &  &  & 0.6\\
\hline
\hspace{1em}non & 115 (97\%) & 85 (97\%) & 30 (100\%) & \\
\hline
\hspace{1em}oui & 3 (2.5\%) & 3 (3.4\%) & 0 (0\%) & \\
\hline
\textbf{diabete.gestationnel.sous.insuline} &  &  &  & >0.9\\
\hline
\hspace{1em}non & 116 (98\%) & 86 (98\%) & 30 (100\%) & \\
\hline
\hspace{1em}oui & 2 (1.7\%) & 2 (2.3\%) & 0 (0\%) & \\
\hline
\textbf{mode.de.sortie} &  &  &  & \textbf{0.023}\\
\hline
\hspace{1em}Décès & 23 (19\%) & 12 (14\%) & 11 (37\%) & \\
\hline
\hspace{1em}Domicile & 41 (35\%) & 33 (38\%) & 8 (27\%) & \\
\hline
\hspace{1em}Transfert & 54 (46\%) & 43 (49\%) & 11 (37\%) & \\
\hline
\textbf{corticotherapie.an} &  &  &  & 0.5\\
\hline
\hspace{1em}Demi dose & 23 (19\%) & 16 (18\%) & 7 (23\%) & \\
\hline
\hspace{1em}Pleine dose & 95 (81\%) & 72 (82\%) & 23 (77\%) & \\
\hline
\textbf{terme} & 28.00 (26.20, 30.30) & 28.35 (26.37, 30.50) & 26.75 (25.02, 28.88) & \textbf{0.024}\\
\hline
\textbf{cause.prematurite} &  &  &  & 0.4\\
\hline
\hspace{1em}spontanée & 66 (56\%) & 45 (51\%) & 21 (70\%) & \\
\hline
\hspace{1em}ARCF & 16 (14\%) & 13 (15\%) & 3 (10\%) & \\
\hline
\hspace{1em}Pré éclampsie & 29 (25\%) & 24 (27\%) & 5 (17\%) & \\
\hline
\hspace{1em}autre & 7 (5.9\%) & 6 (6.8\%) & 1 (3.3\%) & \\
\hline
\textbf{voie.dacc} &  &  &  & 0.4\\
\hline
\hspace{1em}césarienne & 67 (57\%) & 52 (59\%) & 15 (50\%) & \\
\hline
\hspace{1em}VB & 51 (43\%) & 36 (41\%) & 15 (50\%) & \\
\hline
\textbf{poids.naissance} & 965 (764, 1,410) & 998 (798, 1,421) & 820 (716, 1,338) & 0.3\\
\hline
\textbf{audipog} & 52 (18, 75) & 46 (16, 69) & 64 (36, 80) & \textbf{0.022}\\
\hline
\textbf{rciu} &  &  &  & 0.4\\
\hline
\hspace{1em}Macrosomie & 1 (0.8\%) & 1 (1.1\%) & 0 (0\%) & \\
\hline
\hspace{1em}non & 99 (84\%) & 71 (81\%) & 28 (93\%) & \\
\hline
\hspace{1em}RCIU modéré & 12 (10\%) & 10 (11\%) & 2 (6.7\%) & \\
\hline
\hspace{1em}RCIU sévère & 6 (5.1\%) & 6 (6.8\%) & 0 (0\%) & \\
\hline
\textbf{imf} &  &  &  & \textbf{0.012}\\
\hline
\hspace{1em}non & 50 (42\%) & 42 (48\%) & 8 (27\%) & \\
\hline
\hspace{1em}Suspectée & 56 (47\%) & 41 (47\%) & 15 (50\%) & \\
\hline
\hspace{1em}Confirmée & 12 (10\%) & 5 (5.7\%) & 7 (23\%) & \\
\hline
\textbf{reanimation.en.sdn} &  &  &  & \textbf{0.001}\\
\hline
\hspace{1em}intubation & 2 (1.7\%) & 2 (2.3\%) & 0 (0\%) & \\
\hline
\hspace{1em}Intubation & 51 (44\%) & 33 (38\%) & 18 (60\%) & \\
\hline
\hspace{1em}Intubation+Drogues & 6 (5.1\%) & 2 (2.3\%) & 4 (13\%) & \\
\hline
\hspace{1em}non & 4 (3.4\%) & 2 (2.3\%) & 2 (6.7\%) & \\
\hline
\hspace{1em}VNI & 54 (46\%) & 48 (55\%) & 6 (20\%) & \\
\hline
\hspace{1em}Unknown & 1 & 1 & 0 & \\
\hline
\textbf{p.h.au.cordon} & 7.28 (7.23, 7.32) & 7.28 (7.23, 7.32) & 7.30 (7.24, 7.34) & 0.5\\
\hline
\hspace{1em}Unknown & 11 & 7 & 4 & \\
\hline
\textbf{lactates} & 3.20 (2.50, 4.90) & 3.20 (2.50, 4.70) & 2.95 (2.53, 4.85) & 0.8\\
\hline
\hspace{1em}Unknown & 13 & 9 & 4 & \\
\hline
\textbf{intubation} &  &  &  & >0.9\\
\hline
\hspace{1em}non & 4 (3.4\%) & 3 (3.5\%) & 1 (3.3\%) & \\
\hline
\hspace{1em}oui & 112 (97\%) & 83 (97\%) & 29 (97\%) & \\
\hline
\hspace{1em}Unknown & 2 & 2 & 0 \vphantom{3} & \\
\hline
\textbf{nbre.intubations} &  &  &  & \textbf{0.003}\\
\hline
\hspace{1em}≥ 3 & 13 (11\%) & 6 (6.8\%) & 7 (23\%) & \\
\hline
\hspace{1em}1 & 75 (64\%) & 63 (72\%) & 12 (40\%) & \\
\hline
\hspace{1em}2 & 30 (25\%) & 19 (22\%) & 11 (37\%) & \\
\hline
\textbf{in.sur.e} &  &  &  & >0.9\\
\hline
\hspace{1em}non & 111 (96\%) & 82 (95\%) & 29 (97\%) & \\
\hline
\hspace{1em}oui & 5 (4.3\%) & 4 (4.7\%) & 1 (3.3\%) & \\
\hline
\hspace{1em}Unknown & 2 & 2 & 0 \vphantom{2} & \\
\hline
\textbf{lisa} &  &  &  & 0.6\\
\hline
\hspace{1em}non & 113 (97\%) & 83 (97\%) & 30 (100\%) & \\
\hline
\hspace{1em}oui & 3 (2.6\%) & 3 (3.5\%) & 0 (0\%) & \\
\hline
\hspace{1em}Unknown & 2 & 2 & 0 \vphantom{1} & \\
\hline
\textbf{fi.o2} & 0.53 (0.40, 1.00) & 0.50 (0.40, 0.70) & 0.70 (0.50, 1.00) & \textbf{0.011}\\
\hline
\hspace{1em}Unknown & 6 & 6 & 0 & \\
\hline
\textbf{detresse.respiratoire} &  &  &  & 0.5\\
\hline
\hspace{1em}non & 1 (0.8\%) & 1 (1.1\%) & 0 (0\%) & \\
\hline
\hspace{1em}oui & 110 (93\%) & 83 (94\%) & 27 (90\%) & \\
\hline
\hspace{1em}secondaire & 7 (5.9\%) & 4 (4.5\%) & 3 (10\%) & \\
\hline
\textbf{stade.radio.mmh} &  &  &  & \textbf{<0.001}\\
\hline
\hspace{1em}MMH 1 & 23 (19\%) & 23 (26\%) & 0 (0\%) & \\
\hline
\hspace{1em}MMH 2 & 47 (40\%) & 41 (47\%) & 6 (20\%) & \\
\hline
\hspace{1em}MMH 3 & 25 (21\%) & 16 (18\%) & 9 (30\%) & \\
\hline
\hspace{1em}MMH 4 & 23 (19\%) & 8 (9.1\%) & 15 (50\%) & \\
\hline
\textbf{corticotherapie.post.natale} &  &  &  & \textbf{<0.001}\\
\hline
\hspace{1em}non & 94 (81\%) & 78 (91\%) & 16 (53\%) & \\
\hline
\hspace{1em}oui & 22 (19\%) & 8 (9.3\%) & 14 (47\%) & \\
\hline
\hspace{1em}Unknown & 2 & 2 & 0 & \\
\hline
\multicolumn{5}{l}{\rule{0pt}{1em}\textsuperscript{1} n (\%); Median (IQR)}\\
\multicolumn{5}{l}{\rule{0pt}{1em}\textsuperscript{2} Pearson's Chi-squared test; Fisher's exact test; Wilcoxon rank sum test}\\
\end{tabular}
\end{table}

Le terme, la présence d'une IMF, la gravité en salle de naissance
(intubation, réanimation\ldots) ou le score MMH plus grave semblent
influer la dose de surfactant reçue.

\begin{figure}
\centering
\includegraphics{curosurf_stat_files/figure-latex/gterme1-1.pdf}
\caption{Doses de surfactant selon le terme (num)}
\end{figure}

\begin{figure}
\centering
\includegraphics{curosurf_stat_files/figure-latex/gterme2-1.pdf}
\caption{Doses de surfactant selon le terme}
\end{figure}

\begin{figure}
\centering
\includegraphics{curosurf_stat_files/figure-latex/gauddipog1-1.pdf}
\caption{Doses de surfactant selon le score AUDIPOG}
\end{figure}

\begin{figure}
\centering
\includegraphics{curosurf_stat_files/figure-latex/gimf1-1.pdf}
\caption{Doses de surfactant selon la présence d'une IMF}
\end{figure}

\hypertarget{analyse-multivariuxe9e}{%
\subsubsection{Analyse multivariée}\label{analyse-multivariuxe9e}}

On garde pour cette analyse en régression logistique tous les items
ayant une p-value \textless{} 20\% \& bien sûr la dose de corticoïdes
anténatale. Le meilleur modèle sera ensuite recherché par un
step-by-step descendant.

\begin{table}

\caption{\label{tab:multi}Facteurs de risque - Analyse multivariée}
\centering
\begin{tabular}[t]{l|c|c|c}
\hline
**Characteristic** & **OR** & **95\% CI** & **p-value**\\
\hline
\_\_corticotherapie.an\_\_ &  &  & \\
\hline
Demi dose &  &  & \\
\hline
Pleine dose & 1.01 & 0.33, 3.33 & >0.9\\
\hline
\_\_cause.prematurite\_\_ &  &  & \\
\hline
spontanée &  &  & \\
\hline
ARCF & 1.37 & 0.23, 7.36 & 0.7\\
\hline
Pré éclampsie & 1.08 & 0.22, 5.28 & >0.9\\
\hline
autre & 0.42 & 0.02, 3.08 & 0.5\\
\hline
\_\_cut.terme\_\_ &  &  & \\
\hline
prématurité moyenne &  &  & \\
\hline
grande prématurité & 0.09 & 0.01, 0.48 & \_\_0.009\_\_\\
\hline
très grande prématurité & 1.08 & 0.34, 3.66 & 0.9\\
\hline
\_\_imf\_\_ &  &  & \\
\hline
non &  &  & \\
\hline
Suspectée & 1.58 & 0.41, 6.48 & 0.5\\
\hline
Confirmée & 5.44 & 0.92, 36.6 & 0.072\\
\hline
\end{tabular}
\end{table}

\hypertarget{crituxe8res-secondaires}{%
\subsection{Critères secondaires}\label{crituxe8res-secondaires}}

\hypertarget{intervalle-entre-la-derniuxe8re-dose-de-corticothuxe9rapie}{%
\subsubsection{Intervalle entre la dernière dose de
corticothérapie}\label{intervalle-entre-la-derniuxe8re-dose-de-corticothuxe9rapie}}

\emph{Observer l'impact de l'intervalle entre la dernière dose de
corticothérapie anténatale et la naissance sur la sévérité de la MMH.}

Non significatif (p = 0.482).
\includegraphics{curosurf_stat_files/figure-latex/ginter1-1.pdf}

\hypertarget{pruxe9maturituxe9mmh}{%
\subsubsection{Prématurité/MMH}\label{pruxe9maturituxe9mmh}}

\emph{Observer l'influence de la cause de prématurité sur la sévérité de
la MMH.}

Non significatif (p = 0.578). Mais il y a trop de niveaux (tableau 4 *
4) pour l'effectif donc peu fiable.

\begin{figure}
\centering
\includegraphics{curosurf_stat_files/figure-latex/gmmh1-1.pdf}
\caption{Stade MMH selon la cause de prématurité}
\end{figure}

\hypertarget{corticothuxe9rapie-antuxe9natalepostnatale}{%
\subsubsection{Corticothérapie
anténatale/postnatale}\label{corticothuxe9rapie-antuxe9natalepostnatale}}

\emph{Observer l'impact du nombre de doses de corticothérapie anténatale
sur l'usage de corticothérapie postnatale.}

Pas d'influence (p = 0.763).

\begin{figure}
\centering
\includegraphics{curosurf_stat_files/figure-latex/gantepost-1.pdf}
\caption{Corticothérapie postnatale vs anténatale}
\end{figure}

\hypertarget{dysplasie-bronchopulmonaire}{%
\subsubsection{Dysplasie
bronchopulmonaire}\label{dysplasie-bronchopulmonaire}}

\emph{Observer l'impact du nombre de doses de corticothérapie anténatale
sur la prévalence de dysplasie bronchopulmonaire.}

Pas d'influence (p = 1).

\begin{figure}
\centering
\includegraphics{curosurf_stat_files/figure-latex/gdbp-1.pdf}
\caption{Dysplasie bronchopulmonaire \& corticothérapie anténatale}
\end{figure}

\hypertarget{huxe9morragie-intraventriculaire}{%
\subsubsection{Hémorragie
intraventriculaire}\label{huxe9morragie-intraventriculaire}}

\emph{Observer l'impact du nombre de doses de corticothérapie anténatale
sur la sévérité de l'hémorragie intraventriculaire.}

Pas d'influence (p = 0.286).

\begin{figure}
\centering
\includegraphics{curosurf_stat_files/figure-latex/ghiv-1.pdf}
\caption{Hémorragie intraventriculaire vs corticothérapie anté natale}
\end{figure}

\hypertarget{demandes-diverses-hors-protocole}{%
\subsection{Demandes diverses hors
protocole}\label{demandes-diverses-hors-protocole}}

Inborn/outborn ?

Pour les différents germes il n'y a pas assez de cas pour faire le
moindre calcul.

\begin{table}

\caption{\label{tab:xtable1}Tableau 1}
\centering
\begin{tabular}[t]{l|c|c|c|c}
\hline
\multicolumn{2}{c|}{ } & \multicolumn{2}{c|}{Corticoïdes antenatal} & \multicolumn{1}{c}{ } \\
\cline{3-4}
\textbf{Characteristic} & \textbf{Overall}, N = 118 & \textbf{Demi dose}, N = 23 & \textbf{Pleine dose}, N = 95 & \textbf{p-value}\\
\hline
\textbf{voie.dacc} &  &  &  & 0.3\\
\hline
\hspace{1em}césarienne & 67 (57\%) & 11 (48\%) & 56 (59\%) & \\
\hline
\hspace{1em}VB & 51 (43\%) & 12 (52\%) & 39 (41\%) & \\
\hline
\textbf{terme} & 28.00 (26.20, 30.30) & 29.00 (26.20, 32.15) & 28.00 (26.20, 30.00) & 0.5\\
\hline
\textbf{cause.prematurite} &  &  &  & 0.8\\
\hline
\hspace{1em}spontanée & 66 (56\%) & 14 (61\%) & 52 (55\%) & \\
\hline
\hspace{1em}ARCF & 16 (14\%) & 3 (13\%) & 13 (14\%) & \\
\hline
\hspace{1em}Pré éclampsie & 29 (25\%) & 4 (17\%) & 25 (26\%) & \\
\hline
\hspace{1em}autre & 7 (5.9\%) & 2 (8.7\%) & 5 (5.3\%) & \\
\hline
\textbf{poids.naissance} & 965 (764, 1,410) & 1,110 (835, 1,886) & 935 (728, 1,270) & 0.063\\
\hline
\textbf{rciu} &  &  &  & 0.3\\
\hline
\hspace{1em}Macrosomie & 1 (0.8\%) & 1 (4.3\%) & 0 (0\%) & \\
\hline
\hspace{1em}non & 99 (84\%) & 20 (87\%) & 79 (83\%) & \\
\hline
\hspace{1em}RCIU modéré & 12 (10\%) & 2 (8.7\%) & 10 (11\%) & \\
\hline
\hspace{1em}RCIU sévère & 6 (5.1\%) & 0 (0\%) & 6 (6.3\%) & \\
\hline
\textbf{diabete.1} &  &  &  & 0.5\\
\hline
\hspace{1em}non & 115 (97\%) & 22 (96\%) & 93 (98\%) & \\
\hline
\hspace{1em}oui & 3 (2.5\%) & 1 (4.3\%) & 2 (2.1\%) & \\
\hline
\textbf{diabete.gestationnel.sous.insuline} &  &  &  & >0.9\\
\hline
\hspace{1em}non & 116 (98\%) & 23 (100\%) & 93 (98\%) & \\
\hline
\hspace{1em}oui & 2 (1.7\%) & 0 (0\%) & 2 (2.1\%) & \\
\hline
\textbf{p.h.au.cordon} & 7.28 (7.23, 7.32) & 7.24 (7.14, 7.34) & 7.29 (7.24, 7.32) & 0.3\\
\hline
\hspace{1em}Unknown & 11 & 5 & 6 & \\
\hline
\textbf{lactates} & 3.20 (2.50, 4.90) & 4.00 (2.60, 7.20) & 3.10 (2.50, 4.40) & 0.2\\
\hline
\hspace{1em}Unknown & 13 & 6 & 7 & \\
\hline
\textbf{detresse.respiratoire} &  &  &  & >0.9\\
\hline
\hspace{1em}non & 1 (0.8\%) & 0 (0\%) & 1 (1.1\%) & \\
\hline
\hspace{1em}oui & 110 (93\%) & 22 (96\%) & 88 (93\%) & \\
\hline
\hspace{1em}secondaire & 7 (5.9\%) & 1 (4.3\%) & 6 (6.3\%) & \\
\hline
\textbf{imf} &  &  &  & 0.4\\
\hline
\hspace{1em}non & 50 (42\%) & 8 (35\%) & 42 (44\%) & \\
\hline
\hspace{1em}Suspectée & 56 (47\%) & 11 (48\%) & 45 (47\%) & \\
\hline
\hspace{1em}Confirmée & 12 (10\%) & 4 (17\%) & 8 (8.4\%) & \\
\hline
\textbf{mmh\textbackslash{}\_rec} &  &  &  & 0.4\\
\hline
\hspace{1em}MMH 1-2 & 70 (59\%) & 12 (52\%) & 58 (61\%) & \\
\hline
\hspace{1em}MMH 3-4 & 48 (41\%) & 11 (48\%) & 37 (39\%) & \\
\hline
\multicolumn{5}{l}{\rule{0pt}{1em}\textsuperscript{1} n (\%); Median (IQR)}\\
\multicolumn{5}{l}{\rule{0pt}{1em}\textsuperscript{2} Pearson's Chi-squared test; Wilcoxon rank sum test; Fisher's exact test}\\
\end{tabular}
\end{table}

\begin{table}

\caption{\label{tab:xtable3}Tableau 3}
\centering
\begin{tabular}[t]{l|c|c|c|c|c|c}
\hline
\multicolumn{2}{c|}{ } & \multicolumn{3}{c|}{Cause de la prématurité} & \multicolumn{2}{c}{ } \\
\cline{3-5}
\textbf{Characteristic} & \textbf{Overall}, N = 118 & \textbf{spontanée}, N = 66 & \textbf{ARCF}, N = 16 & \textbf{Pré éclampsie}, N = 29 & \textbf{autre}, N = 7 & \textbf{p-value}\\
\hline
\textbf{mmh\textbackslash{}\_rec} &  &  &  &  &  & 0.6\\
\hline
\hspace{1em}MMH 1-2 & 70 (59\%) & 36 (55\%) & 9 (56\%) & 20 (69\%) & 5 (71\%) & \\
\hline
\hspace{1em}MMH 3-4 & 48 (41\%) & 30 (45\%) & 7 (44\%) & 9 (31\%) & 2 (29\%) & \\
\hline
\textbf{nb.dose.curosurf} &  &  &  &  &  & 0.4\\
\hline
\hspace{1em}1 dose & 88 (75\%) & 45 (68\%) & 13 (81\%) & 24 (83\%) & 6 (86\%) & \\
\hline
\hspace{1em}2 doses ou plus & 30 (25\%) & 21 (32\%) & 3 (19\%) & 5 (17\%) & 1 (14\%) & \\
\hline
\textbf{intervalle.cortico.accouchement} &  &  &  &  &  & 0.7\\
\hline
\hspace{1em}< 12 H & 20 (17\%) & 10 (15\%) & 2 (12\%) & 7 (25\%) & 1 (14\%) & \\
\hline
\hspace{1em}entre 12 H et 24H & 18 (15\%) & 13 (20\%) & 1 (6.2\%) & 3 (11\%) & 1 (14\%) & \\
\hline
\hspace{1em}> 24 H & 79 (68\%) & 43 (65\%) & 13 (81\%) & 18 (64\%) & 5 (71\%) & \\
\hline
\hspace{1em}Unknown & 1 & 0 & 0 & 1 & 0 & \\
\hline
\multicolumn{7}{l}{\rule{0pt}{1em}\textsuperscript{1} n (\%)}\\
\multicolumn{7}{l}{\rule{0pt}{1em}\textsuperscript{2} Fisher's exact test}\\
\end{tabular}
\end{table}

\begin{table}

\caption{\label{tab:xtable4}Tableau 4}
\centering
\begin{tabular}[t]{l|c|c|c|c|c}
\hline
\multicolumn{2}{c|}{ } & \multicolumn{3}{c|}{intervalle corticoïde/accouchement)} & \multicolumn{1}{c}{ } \\
\cline{3-5}
\textbf{Characteristic} & \textbf{Overall}, N = 117 & \textbf{< 12 H}, N = 20 & \textbf{entre 12 H et 24H}, N = 18 & \textbf{> 24 H}, N = 79 & \textbf{p-value}\\
\hline
\textbf{mmh\textbackslash{}\_rec} &  &  &  &  & 0.2\\
\hline
\hspace{1em}MMH 1-2 & 69 (59\%) & 12 (60\%) & 7 (39\%) & 50 (63\%) & \\
\hline
\hspace{1em}MMH 3-4 & 48 (41\%) & 8 (40\%) & 11 (61\%) & 29 (37\%) & \\
\hline
\multicolumn{6}{l}{\rule{0pt}{1em}\textsuperscript{1} n (\%)}\\
\multicolumn{6}{l}{\rule{0pt}{1em}\textsuperscript{2} Pearson's Chi-squared test}\\
\end{tabular}
\end{table}

\begin{table}

\caption{\label{tab:xtable5}Tableau 5}
\centering
\begin{tabular}[t]{l|c|c|c|c}
\hline
\multicolumn{2}{c|}{ } & \multicolumn{2}{c|}{Corticoïdes antenataux)} & \multicolumn{1}{c}{ } \\
\cline{3-4}
\textbf{Characteristic} & \textbf{Overall}, N = 118 & \textbf{Demi dose}, N = 23 & \textbf{Pleine dose}, N = 95 & \textbf{p-value}\\
\hline
\textbf{duree.vc} & 2.0 (1.0, 5.0) & 1.0 (1.0, 5.0) & 2.0 (1.0, 5.0) & 0.8\\
\hline
\textbf{duree.vni} & 12 (3, 39) & 4 (2, 36) & 13 (3, 39) & 0.5\\
\hline
\textbf{pneumothorax} &  &  &  & 0.5\\
\hline
\hspace{1em}non & 115 (97\%) & 22 (96\%) & 93 (98\%) & \\
\hline
\hspace{1em}oui & 3 (2.5\%) & 1 (4.3\%) & 2 (2.1\%) & \\
\hline
\textbf{hemorragie.pulmonaire} &  &  &  & \textbf{0.046}\\
\hline
\hspace{1em}non & 111 (96\%) & 19 (86\%) & 92 (98\%) & \\
\hline
\hspace{1em}oui & 5 (4.3\%) & 3 (14\%) & 2 (2.1\%) & \\
\hline
\hspace{1em}Unknown & 2 & 1 & 1 & \\
\hline
\textbf{dbp} &  &  &  & >0.9\\
\hline
\hspace{1em}non & 68 (83\%) & 16 (84\%) & 52 (83\%) & \\
\hline
\hspace{1em}oui & 14 (17\%) & 3 (16\%) & 11 (17\%) & \\
\hline
\hspace{1em}Unknown & 36 & 4 & 32 & \\
\hline
\textbf{hiv} &  &  &  & 0.3\\
\hline
\hspace{1em}non & 73 (66\%) & 12 (60\%) & 61 (67\%) & \\
\hline
\hspace{1em}grade 1 & 11 (9.9\%) & 1 (5.0\%) & 10 (11\%) & \\
\hline
\hspace{1em}grade 2 & 14 (13\%) & 2 (10\%) & 12 (13\%) & \\
\hline
\hspace{1em}grade 3 & 5 (4.5\%) & 2 (10\%) & 3 (3.3\%) & \\
\hline
\hspace{1em}grade 4 & 8 (7.2\%) & 3 (15\%) & 5 (5.5\%) & \\
\hline
\hspace{1em}Unknown & 7 & 3 & 4 & \\
\hline
\textbf{duree.hospitalisation} & 45 (11, 67) & 38 (18, 77) & 45 (11, 65) & 0.7\\
\hline
\hspace{1em}Unknown & 3 & 1 & 2 & \\
\hline
\textbf{ecun.severe} &  &  &  & 0.6\\
\hline
\hspace{1em}non & 112 (95\%) & 23 (100\%) & 89 (94\%) & \\
\hline
\hspace{1em}oui & 6 (5.1\%) & 0 (0\%) & 6 (6.3\%) & \\
\hline
\textbf{deces} &  &  &  & >0.9\\
\hline
\hspace{1em}Décès & 23 (19\%) & 4 (17\%) & 19 (20\%) & \\
\hline
\hspace{1em}Vivant & 95 (81\%) & 19 (83\%) & 76 (80\%) & \\
\hline
\multicolumn{5}{l}{\rule{0pt}{1em}\textsuperscript{1} Median (IQR); n (\%)}\\
\multicolumn{5}{l}{\rule{0pt}{1em}\textsuperscript{2} Wilcoxon rank sum test; Fisher's exact test}\\
\end{tabular}
\end{table}

\hypertarget{technique}{%
\subsection{Technique}\label{technique}}

Les données discrètes ont été présentés en pourcentage puis comparées
par le test du \(\Chi^2\) de Pearson avec correction de Yates si
nécessaire. Les données numériques ont été présentées par leur médiane
\& les quartiles puis comparées par le test non paramétrique de
Wilcoxon.

L'analyse multivarié a été menée en régression logistique. Les
conditions d'utilisation d'une loi binomiale (distribution normale des
résidus) n'étant pas remplies une loi quasi-binomiale a été utilisée.

L'analyse statistique a été réalisée avec le logiciel \textbf{R} (R Core
Team 2022) \& diverses librairies en particulier celles du
\texttt{tidyverse} (Wickham et al. 2019) \& \texttt{epiDisplay}
(Chongsuvivatwong 2018).

\hypertarget{refs}{}
\begin{CSLReferences}{1}{0}
\leavevmode\vadjust pre{\hypertarget{ref-epid}{}}%
Chongsuvivatwong, Virasakdi. 2018. \emph{epiDisplay: Epidemiological
Data Display Package}.
\url{https://CRAN.R-project.org/package=epiDisplay}.

\leavevmode\vadjust pre{\hypertarget{ref-rstat}{}}%
R Core Team. 2022. \emph{R: A Language and Environment for Statistical
Computing}. Vienna, Austria: R Foundation for Statistical Computing.
\url{https://www.R-project.org/}.

\leavevmode\vadjust pre{\hypertarget{ref-tidy}{}}%
Wickham, Hadley, Mara Averick, Jennifer Bryan, Winston Chang, Lucy
D'Agostino McGowan, Romain François, Garrett Grolemund, et al. 2019.
{«~Welcome to the {tidyverse}~»}. \emph{Journal of Open Source Software}
4 (43): 1686. \url{https://doi.org/10.21105/joss.01686}.

\end{CSLReferences}

\end{document}
